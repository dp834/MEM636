\documentclass{article}

\usepackage[margin=.75in]{geometry}
\usepackage{amsmath}
\usepackage{amssymb}
\usepackage[shortlabels]{enumitem}
\usepackage{graphicx}
\usepackage{float}
\usepackage{subcaption}
\usepackage{pgfplots}

\title{Homework 1 \\ MEM 636}
\author{Athrav Joshi, Damien Prieur, Eamon Fitzgerald, Tuote Huang}
\date{}

\begin{document}

\maketitle

\section*{Problem 1}
\emph{For each of the scalar systems:}
\begin{enumerate}[(1)]
\item $\dot{x} = 2 - 3x +x^2$
\item $\dot{x} = -40x + 40x^2 + 10x^3$
\end{enumerate}
\emph{Determine all of the equilibrium points and evaluate their stability.}
\begin{enumerate}[(1)]
\item $\dot{x} = 2 - 3x +x^2$
$$\dot{x} = (x-2)(x-1) = 0$$
$$ x_1 = 1 \qquad x_2 = 2$$
Looking at the plot for $\dot{x}(t)$

\begin{figure}[H]
\begin{subfigure}{.5\textwidth}
    \centering
    \includegraphics[width=.8\linewidth]{{images/plot1_1}.png}
\end{subfigure}
\begin{subfigure}{.5\textwidth}
    \centering
    \includegraphics[width=.8\linewidth]{{images/streamplot1_1}.png}
\end{subfigure}
\end{figure}

We can see that at $x = 2$ the point is attracting from both the left and the right due to the negative slope.
Meanwhile we can see that at $x = 3$ the point is repelling due to the positive slope.

\item $\dot{x} = -40x + 40x^2 + 10x^3$
$$\dot{x} = x(x-2(\sqrt{2}-1))(x+2(\sqrt{2}+1)) = 0$$
$$ x_1 = 0 \qquad x_2 = -2(\sqrt{2}+1) \qquad x_3 = 2(\sqrt{2} -1)$$
Looking at the plot for $\dot{x}(t)$

\begin{figure}[H]
\begin{subfigure}{.5\textwidth}
    \centering
    \includegraphics[width=.8\linewidth]{{images/plot1_2}.png}
\end{subfigure}
\begin{subfigure}{.5\textwidth}
    \centering
    \includegraphics[width=.8\linewidth]{{images/streamplot1_2}.png}
\end{subfigure}
\end{figure}

We can see that at $x = 0$ the point is attracting from both the left and the right due to the negative slope.
Meanwhile we can see that at $x = -2(\sqrt{2}+1)$ and $x = 2(\sqrt{2} -1) $ the points are repelling due to the positive slope.
\end{enumerate}

\newpage
\section*{Problem 2}
\emph{Consider the system}
$$\dot{x}_1 = x_2$$
$$\dot{x}_2 = -x^3_1 - cx_2$$
\begin{enumerate}[(1)]
\item
\emph{Show that the origin is an equilibrium point.}
\newline
$$\dot{x}_1 = 0$$
$$\dot{x}_2 = 0^3 - c0 = 0$$
Both derivatives are zero at the origin so it's an equilibrium point.
\item
\emph{Linearize at the origin and determine if the origin is linearly stable, asymptotically stable.}
\newline
We first compute the Jacobian evaluated $\frac{\partial f_i}{\partial x_i}$ at $x=0$.
Rewriting the system in matrix form
$$\Delta\dot{x}=Ax$$
Where matrix 'A' represents the partial derivatives evaluated at the origin $(0,0)$.
$$\Delta\dot{x}=
\begin{bmatrix}
0 & 1\\
0 & -c
\end{bmatrix}
x
$$
\begin{figure}[H]
\begin{subfigure}{.5\textwidth}
    \centering
    \includegraphics[width=.8\linewidth]{{images/vectorplot2_2(Pos_c)}.png}
    \caption{Positive c}
\end{subfigure}
\begin{subfigure}{.5\textwidth}
    \centering
    \includegraphics[width=.8\linewidth]{{images/vectorplot2_2(Neg_c)}.png}
    \caption{Negative c}
\end{subfigure}
\newline
\begin{subfigure}{.5\textwidth}
    \centering
    \includegraphics[width=.8\linewidth]{{images/streamplot2_2(Pos_c)}.png}
    \caption{Positive c}
\end{subfigure}
\begin{subfigure}{.5\textwidth}
    \centering
    \includegraphics[width=.8\linewidth]{{images/streamplot2_2(Neg_c)}.png}
    \caption{Negative c}
\end{subfigure}
\end{figure}
To find if the origin is stable we need to look at the Lyapunov equation $Q^TP+QA=-P$ where $A$ is our matrix from our linear system.
\newline
$$
Q=
\begin{bmatrix}
1 & 0 \\
0 & 1
\end{bmatrix}
\implies
V(x)=
\begin{bmatrix}
x_1 & x_2
\end{bmatrix}
\begin{bmatrix}
1 & 0 \\
0 & 1
\end{bmatrix}
\begin{bmatrix}
x_1 \\
x_2
\end{bmatrix}
=x_1^2 + x_2^2
$$
Which is positive definite. Find $P$
$$
P=
-(QA+A^TQ)
=
\begin{bmatrix}
0 & -1  \\
-1 & 2c \\
\end{bmatrix}
$$
We can see that $|P| = -1$ i.e. $|P|<0$.
The eigenvalues of $P$ give us
$$\lambda = c \pm \sqrt{c^2-1}$$
Which has eigen values on both sides of the plane
\newpage
\item
\emph{Analyze the nonlinear system and determine if the origin is stable, asymptotically stable. If so, estimate the region of attraction.}
\newline
Take $$V = \frac{1}{2}x_1^4 + x_2^2$$
$$\dot{V}
=
2x_1^3\dot{x}_1 + 2x_2\dot{x}_2
=
2x_1^3x_2 + 2x_2(-x_1^3 - cx_2)
$$
$$
\dot{V}
=
2x_1^3x_2 - 2cx_2^2 -2x_2x_1^3
=
2x_1^3x_2 - 2x_2x_1^3 - 2cx_2^2
$$
$$
\dot{V} =
-2cx_2^2
$$
Now, $\forall c \geq 0$
$$\dot{V}(x) \leq 0$$
Then the origin is globally asymptotically stable (Corollary 2.23).
\newline
But, $\forall c < 0$
$$\dot{V}(x) > 0$$
Now define a region $D_1 = \{x_1, x_2 | x_1^2 +(x_2-1)^2 \leq 1^2\}$
$$\dot{V}(1+\sqrt{1 \pm x_2^2}, x_2) = -2cx_2^2 \quad c<0 \implies \dot{V} > 0 $$
We can see that along the contour we are positive definite.
And everywhere else in the region $x_2 > 0$ so $\dot{V} > 0$
\newline
Then according to Chetaev Instability Theorem (Proposition 2.24) the origin is unstable.

\begin{figure}[H]
\begin{subfigure}{.5\textwidth}
    \centering
    \includegraphics[width=.8\linewidth]{{images/vectorplot2_3(Pos_c)}.png}
    \caption{Positive c}
\end{subfigure}
\begin{subfigure}{.5\textwidth}
    \centering
    \includegraphics[width=.8\linewidth]{{images/vectorplot2_3(Neg_c)}.png}
    \caption{Negative c}
\end{subfigure}
\newline
\begin{subfigure}{.5\textwidth}
    \centering
    \includegraphics[width=.8\linewidth]{{images/streamplot2_3(Pos_c)}.png}
    \caption{Positive c}
\end{subfigure}
\begin{subfigure}{.5\textwidth}
    \centering
    \includegraphics[width=.8\linewidth]{{images/streamplot2_3(Neg_c)}.png}
    \caption{Negative c}
\end{subfigure}
\end{figure}
\end{enumerate}

\newpage
\section*{Problem 3}
\emph{Investigate the stability of the origin, including estimates of the domain of attraction, of the following systems:}
\begin{enumerate}[(1)]
\item $\ddot{x} = x - sat(2x-\dot{x})$
\newline
Let $x_1 = x$ and $x_2 = \dot{x_1}$. We can rewrite the system as
$$
\begin{array}{c}
x_1 = x \\
x_2 = \dot{x}_1
\end{array}
\implies 
\begin{array}{c}
\dot{x}_1 = x_2 \\
\dot{x}_2 = x_1 - sat(2x_1-x_2)
\end{array}
$$
\begin{figure}[H]
\begin{subfigure}{.5\textwidth}
    \centering
    \includegraphics[width=.6\linewidth]{{images/streamplot3_1_combined}.png}
\end{subfigure}
\begin{subfigure}{.5\textwidth}
    \centering
    \includegraphics[width=.6\linewidth]{{images/vectorplot3_1_combined}.png}
\end{subfigure}
\end{figure}
The highlighted region is boundary between the saturation zones, and the horizontal red lines are not actually there since the boundary is linear and continues forever.

If we look near the origin we can remove the saturation term so our region is $-1 \leq 2x-\dot{x} \leq 1 $
$$
\begin{array}{c}
\dot{x}_1 = x_2 \\
\dot{x}_2 = -x_1 -x_2
\end{array}
$$
This system is at equilibrium when $\dot{x}_1=0$ and $\dot{x}_2=0$.
We can see that the origin $(0,0)$ satisfies this condition.
\newline
Let $V(x_1,x_2) = x_1^2 + x_2^2$. This is positive definite.
$$ \dot{V} = 2 x_1 x_2 + 2 x_2 (-x_1 -x_2) = 2x_1x_2 - 2x_1x_2 -2x_2^2 $$
$$ \dot{V} = -2x_2^2 $$
Which is negative semi-definite therefore it's stable, but not necessarily asymptotically stable.
Looking at the contour plot of $V$ below we can see that the level surfaces are bounded by concentric circles.
And since we showed that $\dot{V} \leq 0$.
All trajectories in the largest set bounded by $ -1 \leq 2x_1 - x_2 \leq 1$ tend to the origin (0,0) so it's Asymptotically stable.
\newline
Region of attraction: $ V(x_1,x_2) \leq .45^2$.
Note this number was not computed accurately, to find the exact value we just need to find the minimum distance between the two region bounding lines.
\begin{figure}[H]
\begin{subfigure}{.5\textwidth}
    \centering
    \includegraphics[width=.6\linewidth]{{images/ContourPlot3_1}.png}
\end{subfigure}
\begin{subfigure}{.5\textwidth}
    \centering
    \includegraphics[width=.6\linewidth]{{images/streamplot3_1_region_of_attraction}.png}
\end{subfigure}
\end{figure}


\newpage
\item $\ddot{x} + \dot{x}|\dot{x}| + x - x^3 = 0 \implies \ddot{x} = x^3 - x - \dot{x}|\dot{x}| $
\newline
Let $x_1 = x$ and $x_2 = \dot{x_1}$. Due to the $|\dot{x}|$ we need to solve two systems split alot $\dot{x} = 0$  We can rewrite the system as
$$
\begin{array}{c}
x_1 = x \\
x_2 = \dot{x}_1
\end{array}
\implies 
\begin{array}{cc}
\dot{x}_1 =& x_2 \\
\dot{x}_2 =& x_1^3 - x_1 - x_2|x_2|
\end{array}
$$

\begin{figure}[H]
\begin{subfigure}{.5\textwidth}
    \centering
    \includegraphics[width=.6\linewidth]{{images/streamplot3_2}.png}
\end{subfigure}
\begin{subfigure}{.5\textwidth}
    \centering
    \includegraphics[width=.6\linewidth]{{images/vectorplot3_2}.png}
\end{subfigure}
\end{figure}

This system is at equilibrium when $\dot{x}_1=0$ and $\dot{x}_2=0$.
We can see that the origin $(0,0)$ satisfies this condition.
\newline
Let $V(x_1,x_2) = -\frac{1}{2}x_1^4 + x_1^2 + x_2^2  \implies x_2 = \pm \sqrt{\frac{1}{2}x_1^4 - x_1^2}$. 
\newline
Define a region $\Omega = \{x_1, x_2 | -\sqrt{\frac{1}{2}x_1^4 - x_1^2} < x_2 < \sqrt{\frac{1}{2}x_1^4 - x_1^2} \}$
\newline
The region $\Omega$ is positive definite since the only other zero on the surface is $(0,0)$ and sampling the points between the zeros we find that the points are positive.
Since the surface is continuous the whole region $\Omega$ is positive definite.
\newline
$$ \dot{V} = -2x_1^3x_2 + 2 x_1 x_2 + 2 x_2 (x_1^3 - x_1 - x_2|x_2|) = -x_1^3x_2 + 2x_1x_2 + 2x_1^3x_2 -2x_1x_2 - 2 x_2^2|x_2| $$
$$ \dot{V} = -2x_2^2|x_2| $$
Which is negative semi-definite over our region.
Therefore the origin is stable. We can look at the contour plot of $V$ and see that we have contour $V = .47 | |x_1| < 1$.
Looking at the contour plot of $V$ below we can see that the level surfaces are bounded by concentric circles.
And since we showed that $\dot{V} \leq 0$.
All trajectories in the largest set bounded by $V = .47 | |x_1| < 1$ tend to the origin (0,0) so it's Asymptotically stable for the region.
\begin{figure}[H]
\begin{subfigure}{.5\textwidth}
    \centering
    \includegraphics[width=.6\linewidth]{{images/contourplot3_2}.png}
\end{subfigure}
\begin{subfigure}{.5\textwidth}
    \centering
    \includegraphics[width=.6\linewidth]{{images/plot3_2}.png}
\end{subfigure}
\end{figure}

\end{enumerate}

\end{document}
